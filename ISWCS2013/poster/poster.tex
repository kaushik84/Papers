\documentclass[11pt]{article}
\usepackage{amsmath}
\usepackage{amssymb}
\usepackage{amsfonts}

\usepackage{hyperref}
\usepackage[paperheight=12in,paperwidth=8.5in]{geometry}
\usepackage{helvet}
\renewcommand{\familydefault}{\sfdefault}
\usepackage{siunitx}

\title{Cognitive Relay: Detecting Spectrum Holes in a Dynamic Scenario}
\date{Ilmenau, 28.08.2013}
\author{Ankit Kaushik, Marcus M\"uller, Friedrich K. Jondral \\
        Communications Engineering Lab, Karlsruhe Institute of Technology (KIT) \\
        \{\href{mailto:Ankit.Kaushik@kit.edu}{Ankit.Kaushik}, 
        \href{mailto:Friedrich.Jondral@kit.edu}{Friedrich.Jondral}\}@kit.edu,
        \href{mailto:Marcus.Mueller@student.kit.edu}{Marcus.Mueller@student.kit.edu} }
\begin{document}
  \maketitle
	\section*{Introduction and Motivation}
	%\textbf{Cognitive radio} illustrates a dynamic model that provides human intervention to the underlying radio hardware. \\
	\textbf{Dynamic Spectrum Access} (DSA) is one of the many applications of cognitive radio.\\
	Main tasks for a DSA operating as Secondary User (SU):
	\begin{itemize}
%	\item Realizes secondary access to the primary user (PU) spectrum
	\item Sensing $\rightarrow$ \textit{Learning} and intelligent access $\rightarrow$ \textit{Act}
	\item Avoid interference to the Primary User (PU)
	\end{itemize}
	
	Purpose of this work:
	\begin{itemize}
	\item Realize the cognitive concepts over the hardware.
	\item Use sophisticated algorithms to reduce the time constraints and computational complexity
	\item Demonstrate the radio in a real scenario
	\end{itemize}
	\section*{Scenario}
		\textbf{Cognitive Relay} (CR) is network element of the SU system.
	\begin{itemize}
	\item Supports wireless services for devices operating indoor
	\item Enables dynamic access to increase spectral efficiency
	\end{itemize}
	\section*{System Model}
	Cross-Layer optimization\\
	\begin{itemize}
		\item {
			Receiver model: 
			\begin{align*}
				y[k] = x[k] + w[k]
			\end{align*}	
			Energy Detection 
			\begin{align*}
				t(\textbf{y}) = \frac{1}{K} \sum\limits_{k=0}^{K-1} |y[k]|^2 \mathop{\gtrless}_{\mathcal{H}_1}^{\mathcal{H}_0} \gamma
			\end{align*}	
			\renewcommand{\arraystretch}{1.3}
			\begin{tabular}{p{0.15\columnwidth}p{0.73\columnwidth}} 
			$t(\textbf{y})$  &    \text{Test statistics}, \\ 
			$K$    & \text{  Number of samples,} \\
			$y[k]$ & \text{  Received waveform,} \\
			$x[k]$ & \text{  Transmitted waveform,} \\
			$w[k]$ & \text{  Noise waveform,} \\
			$\gamma$ & \text{ Threshold, determined using constant false alarm}
			\end{tabular} }	
		\item {Learning	\\
			Modelling channel access model as discrete time discrete state Markov Process (Fig.2)
			\begin{itemize}
				\item PU subchannels $N$
				\item Multiband sensing through subchannel scanning
				\item State transitional probabilities $(\alpha, \beta)$		
				\item {Utilization Probability $u$ 
				\begin{align*}
					u &= \mathbb {P}(z(t) = 0) \\
        				\quad &= \frac{\alpha}{\alpha + \beta }
				\end{align*} }
				\end{itemize}
				}			
		\end{itemize}
	
	\section*{Implementation and User Interaction}
	Demonstrator Setup (Fig. Demonstrator) \\
	\begin{center}
		\begin{tabular}{|c|c|c|}
			\hline
			Test scenario &  GSM Channels at \SI{1800}{MHz} \\ \hline
			Hardware platform & USRP N210 \\ \hline
			Multiband sensing &  \\  Software Defined architecture &  GnuRadio  \\  \hline	
		\end{tabular}
	\end{center}

	Analysis (Fig.3)		\\
	\begin{center}
		\begin{tabular}{|c|c|c|}
		\hline
		Detection of spectral holes &  time slots \\ \hline
		Estimation of model parameters $(\widehat{\alpha}, \widehat{\beta)}$ & Maximum Likelihood Estimation \\ \hline
		Channel Ranking & $\textbf{u} = [u^1,u^2,...,u^N]$	 \\ \hline	
		\end{tabular}
	\end{center}
		
		
	\section*{Conclusion and Future Work}	
			
		\begin{itemize}
			\item Cognitive radio implementation in a dynamic scenario
			\item Potential to sense non-contiguous multiple band simultaneous over low cost hardware		
			\item Capable to learn and interact with its environment
			\item {Considering other scenarios such as Overlay Systems
			\begin{itemize}
				\item spatial separation of the PU and SU systems $\rightarrow$ Transmit Power Control
			\end{itemize}}
			\item Cooperation with other the CRs					
		\end{itemize}	
\end{document}
